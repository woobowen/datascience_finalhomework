\documentclass[a4paper,12pt]{article}

% ==================== 基础宏包配置 ====================
\usepackage[UTF8]{ctex}       % 中文支持
\usepackage{geometry}         % 页面布局
\usepackage{graphicx}         % 图片支持
\usepackage{float}            % 图片浮动控制
\usepackage{listings}         % 代码块展示
\usepackage{xcolor}           % 颜色支持
\usepackage{hyperref}         % 超链接
\usepackage{amsmath}          % 数学公式
\usepackage{booktabs}         % 三线表
\usepackage{caption}          % 标题美化
\usepackage{fancyhdr}         % 页眉页脚
\usepackage{titlesec}         % 标题格式
\usepackage{algorithm}        % 算法伪代码
\usepackage{algorithmic}

% ==================== 路径与页面设置 ====================
% 设置图片搜索路径 (使用您指定的绝对路径)
\graphicspath{{C:/Users/lenovo/Desktop/暑期课/大作业/报告图片/}}

% 页面边距
\geometry{left=2.5cm, right=2.5cm, top=2.5cm, bottom=2.5cm}

% 修复页眉高度警告
\setlength{\headheight}{15pt}

% 页眉页脚
\pagestyle{fancy}
\fancyhf{}
\lhead{数据导论课程大作业}
\rhead{KPL 战术分析与 BP 辅助系统}
\cfoot{\thepage}

% 代码块样式设置
\definecolor{codegreen}{rgb}{0,0.6,0}
\definecolor{codegray}{rgb}{0.5,0.5,0.5}
\definecolor{codepurple}{rgb}{0.58,0,0.82}
\definecolor{backcolour}{rgb}{0.96,0.96,0.96}

\lstdefinestyle{mystyle}{
    backgroundcolor=\color{backcolour},   
    commentstyle=\color{codegreen},
    keywordstyle=\color{magenta},
    numberstyle=\tiny\color{codegray},
    stringstyle=\color{codepurple},
    basicstyle=\ttfamily\footnotesize,
    breakatwhitespace=false,         
    breaklines=true,                 
    captionpos=b,                    
    keepspaces=true,                 
    numbers=left,                    
    numbersep=5pt,                  
    showspaces=false,                
    showstringspaces=false,
    showtabs=false,                  
    tabsize=4,
    frame=single
}
\lstset{style=mystyle}

% ==================== 文档正文 ====================
\begin{document}

% ------------------- 封面页 -------------------
\begin{titlepage}
    \centering
    \vspace*{1cm}
    
    % 使用主页面截图作为封面图
    \includegraphics[width=0.85\textwidth]{主页面.png} 
    
    \vspace{2cm}
    {\Huge \bfseries 基于多源数据融合的 \\[0.5em] KPL 战术分析与 BP 辅助系统 \par}
    
    \vspace{1.5cm}
    {\Large \bfseries 数据导论课程大作业报告 \par}
    
    \vspace{3cm}
    
    \begin{table}[h]
        \centering
        \Large
        \begin{tabular}{rl}
            \textbf{汇报人:} & 吴博闻 \\
            \textbf{时\quad 间:} & 2026年1月 \\
            \textbf{技术栈:} & Python, SQL Server, RAG, Collaborative Filtering \\
        \end{tabular}
    \end{table}
    
    \vfill
\end{titlepage}

% ------------------- 摘要 -------------------
\begin{abstract}
    随着移动电竞产业的爆发式增长,职业联赛(KPL)产生的数据量呈指数级上升。然而,现有的数据工具多停留在赛后静态统计层面,缺乏对赛中 BP(Ban/Pick)博弈的实时决策支持。本项目构建了一个融合多源数据的战术分析系统。系统底层采用 SQL Server 存储海量异构数据,通过 Python 实现 ETL 管道与特征工程;核心算法层创新性地结合了基于协同过滤(Item-Based CF)的推荐引擎与基于 RAG(检索增强生成)的大模型战术推理;前端采用多线程 GUI 技术实现流畅交互。实验表明,该系统能够有效从非结构化和半结构化数据中提取高价值战术特征,为普通玩家和分析师提供“感性解说”与“理性推荐”双重决策支持。
    
    \textbf{关键词:} 数据挖掘,协同过滤,RAG,特征工程,KPL数据分析
\end{abstract}

\newpage
% ------------------- 目录 -------------------
\tableofcontents
\newpage

% ------------------- 第一章:项目背景与架构 -------------------
\section{项目概况}

\subsection{行业痛点与项目目标}
当前电竞数据分析领域主要面临两大痛点:
\begin{enumerate}
    % 【修复点】这里将 & 转义为 \&
    \item \textbf{数据体量大且杂(Data Variety \& Volume)}:比赛产生的数据涵盖经济曲线(时间序列)、英雄交互(图数据)、BP博弈(序列数据),普通用户难以从中提取有效信息。
    \item \textbf{缺乏实时决策支持(Real-time Decision Making)}:现有工具无法在 BP 阶段提供基于博弈论的实时战术建议,导致数据与应用场景割裂。
\end{enumerate}

本项目旨在打通“数据获取-清洗-建模-应用”的全链路,构建一个能“读懂数据”并提供“实时建议”的智能助手。

\subsection{系统整体架构}
本系统采用经典的 MVC 分层架构设计,实现了数据存储、业务逻辑与前端展示的解耦:
\begin{enumerate}
    \item \textbf{数据层 (Data Layer)}:基于 SQL Server,存储 \texttt{MatchData}(原始比赛流)、\texttt{team\_info}(战队画像)及英雄统计表。
    \item \textbf{逻辑层 (Logic Layer)}:使用 Python (Pandas/PyODBC) 进行数据清洗、特征聚合,并运行协同过滤算法核心。
    \item \textbf{智能层 (Intelligence Layer)}:集成豆包 (Doubao) 大模型 API,基于 RAG 架构实现自然语言生成。
    \item \textbf{表现层 (Presentation Layer)}:基于 wxPython 构建多线程 GUI 界面。
\end{enumerate}

% ------------------- 第二章:数据工程(核心) -------------------
\section{数据工程与底层逻辑}

数据工程是本项目的基础。我们处理的数据具有多源、异构、噪声大的特点,因此构建鲁棒的 ETL(Extract-Transform-Load)管道至关重要。

\subsection{数据库设计与ER模型}
为了支持复杂的关联查询,我们设计了星型模式(Star Schema)的数据库结构:
\begin{itemize}
    \item \textbf{事实表 (Fact Table)}:\texttt{MatchData}。存储细粒度的单场比赛记录,包含比赛ID、日期、蓝红方队伍、胜负结果以及 10 个英雄的 Pick 序列和 8 个 Ban 序列。
    \item \textbf{维度表 (Dimension Tables)}:
        \begin{itemize}
            \item \texttt{team\_info}:记录战队的宏观画像,如胜率、场均击杀(衡量进攻性)、场均死亡(衡量容错率)、分均经济(衡量运营能力)。
            \item \texttt{TeamHeroBan/Choose}:记录各战队对特定英雄的偏好统计。
        \end{itemize}
\end{itemize}

\subsection{数据清洗与噪声处理}
在真实场景中,用户输入的战队名称往往存在不规范(如空格、大小写、别名)。在 \texttt{team\_analysis.py} 中,我们实现了字符串归一化逻辑:

\begin{lstlisting}[language=Python, caption=数据清洗与模糊匹配算法]
# 引用自 team_analysis.py
def get_detailed_team_data(team_name):
    # 字符串归一化:去除空格,转换为标准格式
    clean_name = team_name.replace(' ', '')
    search_param = f"%{clean_name}%"

    # 使用 SQL LIKE 算子进行模糊匹配,增强系统鲁棒性
    cursor.execute("SELECT * FROM team_info WHERE REPLACE(team, ' ', '') LIKE ?", 
                   (search_param,))
\end{lstlisting}

此外,针对数据缺失(Missing Values)的情况,我们在 SQL 查询层和 Python 处理层均加入了空值填充(Imputation)策略,例如将缺失的胜率默认为 50\% 或标记为“暂无数据”,防止下游算法崩溃。

\subsection{特征工程:从数据到画像}
为了量化战队的战术风格,我们进行了深度的特征提取:
\begin{itemize}
    \item \textbf{本命英雄提取 (Signature Pick)}:利用 SQL 窗口函数与聚合算法,按 \texttt{ChooseCount} 降序排列,提取 Top-K 英雄,代表战队的“体系核心”。
    \item \textbf{惧怕特征提取 (Fear Factor)}:统计 \texttt{BanCount},识别战队最不希望遇到的对手英雄,代表战队的“体系短板”。
\end{itemize}

\begin{figure}[H]
    \centering
    \includegraphics[width=0.9\textwidth]{战队比赛数据.png}
    \caption{系统底层的原始比赛数据,经过ETL处理后生成特征}
\end{figure}

% ------------------- 第三章:核心算法实现 -------------------
\section{AI 核心与算法实现}

本项目创新性地采用了“双引擎”智能决策架构,即 **基于统计学的协同过滤** 与 **基于生成式 AI 的 RAG** 深度融合。

\subsection{算法一:基于协同过滤 (Item-Based CF) 的推荐引擎}
为了在 BP 阶段提供理性的英雄推荐,我们自主研发了基于 Item-Based Collaborative Filtering 的推荐算法。该算法的核心假设是:“如果英雄 A 和英雄 B 经常在获胜对局中同时出现,则它们具有高协同性”。

\subsubsection{矩阵构建与向量化}
首先,我们将存储在 SQL 中的长尾交易数据(Transaction Data)转化为稀疏矩阵(Sparse Matrix):
\begin{itemize}
    \item \textbf{共现矩阵 (Co-occurrence Matrix)}:$C_{ij}$ 表示英雄 $i$ 和英雄 $j$ 同时出场的次数。
    \item \textbf{胜场矩阵 (Win Matrix)}:$W_{ij}$ 表示英雄 $i$ 和英雄 $j$ 同时出场且获胜的次数。
\end{itemize}

\subsubsection{相似度计算与评分公式}
不同于传统的余弦相似度,我们设计了一种**“净胜率提升”**评分公式,以剥离英雄自身的基础胜率(User Bias),仅保留“搭配”带来的增益:

\[ Score(H_{target} | H_{selected}) = \frac{W(H_{target} \cap H_{selected})}{C(H_{target} \cap H_{selected})} - P(Win | H_{target}) \]

在 \texttt{bp\_engine.py} 中的实现逻辑如下:

\begin{lstlisting}[language=Python, caption=协同矩阵构建与评分计算]
# 遍历历史数据构建矩阵
for h1 in heroes_list:
    for h2 in heroes_list:
        games = pair_counts.loc[h1, h2]
        if games >= min_games:
            # 计算联合胜率
            joint_win_rate = pair_wins.loc[h1, h2] / games
            # 减去 h1 的基础胜率,提取“净协同值”
            self.synergy_matrix.loc[h1, h2] = joint_win_rate - self.hero_win_rates[h1]
\end{lstlisting}

该算法能够有效发现潜在的强关联组合(如“大乔+老夫子”),而非仅仅推荐高胜率的超模英雄。

\subsection{算法二:基于 RAG 的生成式决策架构}
传统的 LLM(大语言模型)在垂直领域常出现“幻觉”。本项目采用 **RAG(检索增强生成)** 架构解决此问题。

\subsubsection{动态上下文注入}
系统首先通过 SQL 检索出特定战队的结构化画像(如“场均击杀 10.0”、“高频禁用大乔”),然后将这些**真实数据(Context)**动态注入到 System Prompt 中。

\subsubsection{思维链 (Chain-of-Thought) 提示工程}
我们设计了复杂的 Prompt 结构,引导 AI 进行分步推理:
\begin{enumerate}
    \item \textbf{角色设定}:顶级战术分析师。
    \item \textbf{数据解读}:结合场均击杀判断是“打架队”还是“运营队”。
    \item \textbf{逻辑推演}:基于常用英雄分析战术体系。
    \item \textbf{结论输出}:给出具体的 BP 克制建议。
\end{enumerate}

\begin{lstlisting}[language=Python, caption=RAG 提示词构建 (team\_analysis.py)]
system_prompt = """
你是一名KPL王者荣耀职业联赛的顶级战术分析师。
请严格按照以下格式输出:
【战队风格与体系】结合场均击杀(高则为打架队)和常用英雄分析...
【如何克制】针对他们“最常使用”的英雄,对手应该禁用谁?
"""
user_message = f"分析目标:{team_name}\n数据详情:\n{data_context}"
\end{lstlisting}

\subsection{多线程与系统性能优化}
由于 LLM API 的响应通常有 1-2 秒的延迟,为了防止 GUI 界面在计算时“假死”,我们在 \texttt{game.py} 中引入了多线程机制:

\begin{lstlisting}[language=Python, caption=多线程异步调用]
# 启动子线程处理 AI 请求,主线程继续响应 UI 事件
threading.Thread(target=self.call_ai_and_update, 
                 args=(prompt, current_team_picks)).start()
\end{lstlisting}

% ------------------- 第四章:功能模块展示 -------------------
\section{全功能模块展示与分析}

\subsection{实时 BP 模拟与智能博弈}
这是本系统的核心亮点。用户在模拟器中每进行一次选择,后台的 CF 引擎和 RAG 引擎会同时工作:
\begin{itemize}
    \item \textbf{右侧上方(AI 解说)}:提供感性的战术评价(如“阵容偏向前期进攻”)。
    \item \textbf{右侧下方(大数据推荐)}:提供理性的胜率预测(如“建议搭配大司命,胜率提升 12\%”)。
\end{itemize}

\begin{figure}[H]
    \centering
    \includegraphics[width=0.95\textwidth]{BP模拟器过程页.png}
    \caption{实时 BP 模拟界面:展示了“AI解说”与“大数据推荐”的双重输出}
\end{figure}

\subsection{多维度数据透视}
系统支持从“战队”和“位置”两个维度进行深度数据挖掘。

\subsubsection{战队维度分析}
通过可视化图表,用户可以直观看到战队的“本命英雄”(高频选择)与“惧怕点”(高频禁用)。

\begin{figure}[H]
    \centering
    \begin{minipage}{0.48\textwidth}
        \centering
        \includegraphics[width=\linewidth]{战队常选择英雄.png}
        \caption{战队体系核心(常选)}
    \end{minipage}
    \hfill
    \begin{minipage}{0.48\textwidth}
        \centering
        \includegraphics[width=\linewidth]{战队常禁用英雄.png}
        \caption{战队体系短板(常禁)}
    \end{minipage}
\end{figure}

\subsubsection{位置维度分析}
针对不同分路的玩家,系统通过 SQL 的 \texttt{WHERE position = ?} 聚合查询,提供细分的高胜率英雄推荐。

\begin{figure}[H]
    \centering
    \includegraphics[width=0.6\textwidth]{位置高胜率英雄.png}
    \caption{对抗路高胜率英雄推荐:基于全服大数据的筛选}
\end{figure}

\subsection{AI 深度战术报告}
利用 RAG 技术,系统能一键生成结构化的战队分析报告,包含风格、短板及克制策略,字数约 300 字,逻辑清晰。

\begin{figure}[H]
    \centering
    \includegraphics[width=0.95\textwidth]{战队特点分析.png}
    \caption{AI 生成的深度战术分析报告,准确识别出 AG 超玩会的“进攻体系”}
\end{figure}

\subsection{赛后复盘}
BP 结束后,系统会自动触发全盘分析,评价双方最终阵容的优劣势。

\begin{figure}[H]
    \centering
    \includegraphics[width=0.95\textwidth]{BP模拟器总结页.png}
    \caption{赛后阵容 AI 最终复盘}
\end{figure}

% ------------------- 第五章:总结与展望 -------------------
\section{未来展望与总结}

\subsection{未来展望}
本项目已完成了从原始数据存储到智能决策分析的完整闭环,但在算法深度上仍有探索空间:
\begin{enumerate}
    \item \textbf{推荐算法的深度优化}:目前使用的 Item-Based CF 在面对极度稀疏数据(冷门英雄组合)时可能存在偏差。未来计划引入 \textbf{矩阵分解 (Matrix Factorization, e.g., SVD++)} 技术,挖掘潜在特征向量。
    \item \textbf{强化学习 (Reinforcement Learning)}:构建一个基于 RL 的虚拟对战环境,让 AI 进行左右互搏 (Self-Play)。通过模拟数万场 BP 对局,利用策略梯度 (Policy Gradient) 算法自动学习最优 BP 策略,实现从“辅助决策”到“自动决策”的进化。
\end{enumerate}

\subsection{总结}
本项目综合运用了 SQL 数据库技术、Python 数据分析栈、生成式 AI (LLM) 以及经典推荐算法,构建了一个功能完备的 KPL 战术辅助系统。系统不仅解决了海量电竞数据“看不懂”的痛点,更通过“双引擎”架构实现了数据价值的深度挖掘与实时应用,在数据科学应用领域具有较高的实践意义。

\end{document}